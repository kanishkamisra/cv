\documentclass[11pt]{article}

\usepackage[margin=1in]{geometry}
\usepackage{tabularx}
\usepackage{textcomp}
\usepackage{longtable}
\usepackage{url}
\usepackage[dvipsnames]{xcolor}
\usepackage{titlesec}
\usepackage{xifthen}
% \usepackage{sectsty}
% \usepackage{palatino}

\definecolor{darkblue}{HTML}{07689f}
\definecolor{darkred}{HTML}{931a25}

\usepackage[colorlinks = true,
           linkcolor = darkblue,
           urlcolor  = darkblue,
           citecolor = darkblue,
           anchorcolor = darkblue]{hyperref}

% \allsectionsfont{\sffamily}
% \renewcommand{\familydefault}{\sfdefault}
% \usepackage{libertine}
% \usepackage{palatino}

\titleformat{\section}{\normalfont\Large\bfseries}{\thesection}{1em}{}[{\titlerule[0.5pt]}]
% \titleformat{\section}{\large\bfseries}{\thesection}{0.5em}{}[\titlerule]
% \titlespacing{\subsection}{0pt}{-3ex}{4ex}

\setlength{\titlewidth}{\textwidth}

\newcommand{\link}[1]{[\href{#1}{\texttt{link}}]}
\newcommand{\poster}[1]{[\href{#1}{\texttt{poster}}]}
\newcommand{\preprint}[1]{[\href{#1}{\texttt{preprint}}]}
\newcommand{\github}[1]{[\href{#1}{\texttt{github}}]}

\begin{document}
\begin{center}
\textsc{\LARGE Kanishka Misra} \\
\vspace{3mm}
\textit{PhD candidate interested in Natural Language Understanding and Cognitive Science}\\
% Department of Computer and Information Technology, Purdue University\\
\textbf{Email:} \href{mailto:kmisra@purdue.edu}{\texttt{kmisra@purdue.edu}} \hspace{1em}
\textbf{Website:} \url{https://kanishka.website/}\\
\textbf{Last Updated:} \today
\end{center}

% \section*{Contact}
% \begin{tabularx}{\textwidth}{@{}p{.1\textwidth} X}
% % \textbf{Address:} & Department of CIT \\
% % & University of Chicago \\
% % & 1115 E. 58th Street \\
% % & Chicago, IL 60637 \\
% \textbf{Email:} & \texttt{kmisra@purdue.edu} \\
% \textbf{Website:} & \url{https://kanishka.xyz/}\\
% \end{tabularx}

% \section*{Appointments}
% \begin{tabularx}{\textwidth}{@{}p{.15\textwidth} X}
% 2019--present & Assistant Professor, Dept.~of Linguistics, The University of Chicago \\
% 2018--2019 & Research Assistant Professor, Toyota Technological Institute at Chicago
% \end{tabularx}


\section*{Education}
\begin{tabularx}{\textwidth}{@{}p{\textwidth}l}
\textbf{Purdue University, West Lafayette}\\
Ph.D. in Natural Language Understanding, 2018--\textit{present}\\
% Applied Knowledge Representation and Language Understanding (AKRaNLU) Lab\\
Close collaboration with Allyson Ettinger (UChicago Linguistics)\\
\textbf{Dissertation:} \textit{On Semantic Cognition, Inductive Generalization, and Language Models}\\
% \textbf{Research Interests:} Concepts and categories in Language models, Lexical Semantics, Inductive Reasoning from Text, NLP Model evaluation.\\
% With Certificate in Neuroscience and Cognitive Science \\
\textbf{Advisor:} Julia Taylor Rayz\\\\
% Dissertation: \emph{Relating lexical and syntactic processes in language: Bridging research in humans and machines.} \\\\
\textbf{Purdue University, West Lafayette}\\
M.S. in Natural Language Understanding, 2020, \textbf{GPA:} 4.0\\
% With Certificate in Neuroscience and Cognitive Science \\
\textbf{Thesis:}~\textit{Exploring Lexical Sensitivities in Word Prediction Models: A case study on BERT} [\href{https://hammer.figshare.com/articles/thesis/Exploring_Lexical_Sensitivities_in_Word_Prediction_Models_A_case_study_on_BERT/13308830}{\texttt{link}}]\\
\textbf{Advisor:} Julia Taylor Rayz\\
\textbf{Note:} Work performed alongside requirements for Ph.D.\\\\

\textbf{Purdue University, West Lafayette}\\
B.S. \textit{with distinction}. Computer Information Technology, 2014--2018, \textbf{GPA:} 3.72\\
Minor in Statistics
% \textbf{Advisor:} Julia Taylor Rayz\\\\
\end{tabularx}

\section*{Fellowships and Assistantships}
\vspace{-.5em}
\renewcommand*{\arraystretch}{1.5}
\begin{tabularx}{\textwidth}{@{}p{.15\textwidth} X}
2022--present & Purdue Graduate Student Mentoring Fellow. Selected to understand and improve the advising relationship between faculty and students at Purdue University. \textbf{Award:} \$5{,}000 in research and travel funds.\\
2021--present & Research Assistantship funded through NSF EAGER Grant number 2039605. \textbf{Title:} \emph{AI-based Humor-integrated Social Engineering Training}. \textbf{Contribution:} Co-wrote the ``Technical Contribution'' section, and served as key personnel. \textbf{PI:} Julia Taylor Rayz, \textbf{Co-PI:} Ida B. Ngambeki \\
2018--2019 & Purdue Research Foundation (PRF) Fellowship. \textbf{Title:} \textit{Computational Analysis of Online Predatory Texts}. \textbf{Contribution:} Wrote the grant in its entirety. \textbf{Mentor:} Julia Taylor Rayz.\\
\end{tabularx}

\section*{Peer-reviewed Publications}
\vspace{-1.5em}
\renewcommand*{\arraystretch}{1.5}
\begin{longtable}{p{.05\textwidth}  p{.90\textwidth} }
2022 & \textbf{Kanishka Misra}, Allyson Ettinger, Julia Taylor Rayz. A Property Induction Framework for Neural Language Models. \textit{(under review at CogSci 2022)}\\
2022 & \textbf{Kanishka Misra}. On Semantic Cognition, Inductive Generalization, and Language Models. \textit{AAAI 2022 Doctoral Consortium}, Vancouver, Canada. \preprint{https://arxiv.org/abs/2111.02603}\\
2021 & \textbf{Kanishka Misra}, Allyson Ettinger, Julia Taylor Rayz. Do Language Models learn typicality judgments from text? \textit{43rd Annual Conference of the Cognitive Science Society.} (\textbf{\textit{Oral Presentation}}; 14\% acceptance rate) \preprint{https://arxiv.org/abs/2105.02987}\\
2021 & \textbf{Kanishka Misra}, Julia Taylor Rayz. Finding fuzziness in Neural Network models of Language Processing. \textit{Annual Meeting of the North American Fuzzy Information Processing Society 2021.} \textbf{(Honorable Mention for Best Student Paper)}. \preprint{http://kanishka.xyz/papers/nafips21.pdf}\\
2020 & \textbf{Kanishka Misra}, Allyson Ettinger, Julia Taylor Rayz. Exploring BERT's Sensitivity to Lexical Cues using Tests from Semantic Priming. \textit{Findings of the Association for Computational Linguistics: EMNLP 2020}. 
\link{http://dx.doi.org/10.18653/v1/2020.findings-emnlp.415}\\
2020 & Qingyuan Hu, Yi Zhang, \textbf{Kanishka Misra}, Julia Taylor Rayz. Exploring Lexical Irregularities in Hypothesis-Only Models of Natural Language Inference.  \textit{2020 IEEE 19th International Conference on Cognitive Informatics \& Cognitive Computing (ICCI* CC)}. \link{https://arxiv.org/abs/2101.07397}\\
2020 & \textbf{Kanishka Misra}, Julia Taylor Rayz. An Approximate Perspective on Word Prediction in Context: Ontological Semantics meets BERT. \textit{Annual meeting of the North American Fuzzy Information Processing Society 2020} \preprint{https://kanishka.xyz/papers/nafips.pdf}\\
2019 & \textbf{Kanishka Misra}, Hemanth Devarapalli, Tatiana Ringenberg, Julia Taylor Rayz. Authorship Analysis of Online Predatory Conversations using Character Level Convolution Neural Networks. \textit{2019 IEEE International Conference on Systems, Man and Cybernetics (SMC).}, Bari, Italy. \link{https://doi.org/10.1109/SMC.2019.8914323}\\
2019 & Tatiana Ringenberg, \textbf{Kanishka Misra}, Julia Taylor Rayz. Not So Cute but Fuzzy: Estimating Risk of Sexual Predation in Online Conversations. \textit{2019 IEEE International Conference on Systems, Man and Cybernetics (SMC).}, Bari, Italy. \textbf{(joint first author)} \link{https://doi.org/10.1109/SMC.2019.8914528}\\
2019 & Qiaofei Ye, \textbf{Kanishka Misra}, Hemanth Devarapalli, Julia Taylor Rayz. A Sentiment Based Non-Factoid Question-Answering Framework. \textit{2019 IEEE International Conference on Systems, Man and Cybernetics (SMC).}, Bari, Italy. \link{https://doi.org/10.1109/SMC.2019.8913898}\\
2019 & \textbf{Kanishka Misra}, Hemanth Devarapalli, Julia Taylor Rayz. Measuring the Influence of L1 on Learner English Errors in Content Words within Word Embedding Models. \textit{17th International Conference on Cognitive Modelling 2019}., Montréal, Canada. \link{https://kanishka.xyz/papers/iccm.pdf}\\
2019 & Tatiana Ringenberg, \textbf{Kanishka Misra}, Kathryn C. Seigfried-Spellar, Julia Taylor Rayz. Exploring Automatic Identification of Fantasy-Driven and Contact-Driven Sexual Solicitors. \textit{2019 Third IEEE International Conference on Robotic Computing (IRC).}, Naples, Italy. \link{https://doi.org/10.1109/IRC.2019.00110}\\
2019 & Kathryn C Seigfried-Spellar, Marcus K Rogers, Julia T Rayz, Shih-Feng Yang, \textbf{Kanishka Misra}, Tatiana Ringenberg. Chat analysis triage tool: Differentiating contact-driven vs.~fantasy-driven child sex offenders. \textit{Forensic Science International, 2019}. \link{https://doi.org/https://doi.org/10.1016/j.forsciint.2019.02.028}
\end{longtable}

% \section*{Peer-reviewed Thesis Proposals and Abstracts}
\section*{Peer-reviewed Abstracts}
\vspace{-1.5em}
% \textbf{Thesis Proposals}
% \renewcommand*{\arraystretch}{1.5}
% \begin{longtable}{p{.05\textwidth}  p{.90\textwidth} }
% 2021 & \textbf{Kanishka Misra}. On Semantic Cognition, Inductive Generalization, and Language Models. \textit{AAAI 2022 Doctoral Consortium}, Vancouver, Canada. \preprint{https://arxiv.org/abs/2111.02603}
% \end{longtable}
% \noindent
% \textbf{Abtracts}
\begin{longtable}{p{.05\textwidth}  p{.90\textwidth} }
2020 & \textbf{Kanishka Misra}, Allyson Ettinger, Julia Taylor Rayz. Exploring BERT's lexical relations using Semantic Priming. \textit{CogSci 2020} \poster{https://kanishka.xyz/posters/cogsci20.pdf} \link{https://cogsci.mindmodeling.org/2020/papers/0440/index.html}\\
2019 & \textbf{Kanishka Misra}, Hemanth Devarapalli, Julia Taylor Rayz.
L1 Influence on Content Word errors in Learner English Corpora: Insights from Distributed Representation of Words. \textit{CogSci 2019}, Montréal, Canada. \poster{https://kanishka.xyz/posters/cogsci19.pdf} \link{https://cogsci.mindmodeling.org/2019/papers/0626/index.html}
\end{longtable}

% \section*{Presentations}
% \vspace{-1em}
% \begin{longtable}{p{.05\textwidth}  p{.95\textwidth} }
% 2020 & (postponed due to COVID-19) \emph{``Understanding'' and prediction in language: perspectives from AI and human cognition.} Invited talk: University of California, Irvine Summer School on Computational Cognitive Modeling for Language, Irvine, CA.\\
% \end{longtable}

\section*{Honors and Awards}
\vspace{-1.5em}
\renewcommand*{\arraystretch}{1.5}
\begin{longtable}{p{.05\textwidth}  p{.90\textwidth} }
2022 & \textbf{Fellow}, \textit{Purdue Graduate Student Mentoring Fellows Program}. \textbf{Amount:} \$5{,}000 in research funds.\\
2021 & \textbf{Honorable Mention for Best Student Paper}, \textit{North American Fuzzy Information Processing Society.} \textbf{Amount:} \$100.\\
2019 & \textbf{Holistic Safety and Security Research Travel Grant}, \textit{Purdue Polytechnic Institute.} \textbf{Amount:} \$500.\\
2019 & \textbf{CIT Research Travel Grant Award}, \textit{Purdue CIT.} \textbf{Amount:} \$1200 (CogSci 2019), \$600 (IEEE-SMC 2019).\\
2019 & \textbf{Best HSS Poster Presentation}, \textit{CERIAS Symposium.} Award presented by committee on Holistic Safety and Security (HSS) research impact area. \link{https://polytechnic.purdue.edu/office-of-research/impact-areas/holistic-safety-and-security/cerias-poster-session}.\\
2019 & \textbf{Conference Travel Award}, \textit{Chicago R Unconference}. \textbf{Amount:} \$150.\\
2018 & \textbf{PRF Fellowship}, \textit{Purdue Research Foundation}. Covered two semesters worth of graduate school, in addition to stipend.\\
2018 & \textbf{Best Poster Award - PPI}, \textit{Purdue Office of Undergraduate Research Expo.} \textbf{Amount:} \$250. \link{https://www.purdue.edu/undergrad-research/conferences/spring/archive/past-winners.php}\\
2018 & \textbf{Research Scholarship}, \textit{Purdue Office of~Undergraduate Research.} \textbf{Amount:} \$500.\\
2017 & \textbf{First Place}. \textit{Indy Civic Hackathon}. \textbf{Amount:} \$2000 split across 4 team members.
\end{longtable}

\renewcommand*{\arraystretch}{1}
\section*{Teaching}
\begin{tabularx}{\textwidth}{@{}p{.95\textwidth}l}
\large \textbf{Teaching Assistant - \textit{Database Fundamentals} (CNIT 272)}\\
\textbf{Timeline:} Fall 2019, Spring 2020, Fall 2020\\
\textbf{Course Professor:} Dr. Dawn D. Laux\\
Developed lecture videos and taught fundamentals of relational databases and \texttt{SQL} to \underline{three} lab sections ($\approx$ 70 students on average across three semesters).\\
\textbf{Instructor Rating:} 4.8 (on average across three semesters)\\\\

\large \textbf{Guest Lecturer - \textit{Natural Language Technologies} (CNIT 519)}\\
\textbf{Timeline:} Fall 2019, Fall 2020, Spring 2022\\
\textbf{Course Professor:} Dr. Julia Taylor Rayz\\
- Two lectures on Neural Network models of Natural Language Processing\\
- Developing two assignments on neural networks and language models.
\end{tabularx}

% \section*{Industry Experience}
% \begin{tabularx}{\textwidth}{@{}p{.95\textwidth}l}
% \large \textbf{Data Scientist Intern - \textit{Perscio.} Indianapolis}\\
% \textbf{Timeline:} Fall 2019, Spring 2020, Fall 2020\\
% \textbf{Course Professor:} Dr. Dawn D. Laux\\
% Developed lecture videos and taught fundamentals of relational databases and \texttt{SQL} to \underline{three} lab sections ($\approx$ 70 students on average across three semesters).\\
% \textbf{Instructor Rating:} 4.8 (on average across three semesters)\\\\

% \large \textbf{Volunteer Lecturer - \textit{Natural Language Technologies} (CNIT 58101NLT)}\\
% \textbf{Timeline:} Fall 2019, Fall 2020\\
% \textbf{Course Professor:} Dr. Julia Taylor Rayz\\
% - Two lectures on Neural Network models of Natural Language Processing\\
% - Developing two assignments on neural networks and language models.
% \end{tabularx}

% \renewcommand*{\arraystretch}{1.5}

\section*{Work Experience}
\vspace{-1em}
\begin{longtable}{p{.15\textwidth} p{.85\textwidth}}
Summer 2021 & \begin{tabular}[c]{p{.80\textwidth}}\large\textbf{NLP Engineering/Research Intern - \textit{Pythonic AI}}\\Integrating Medical Knowledge into Language Models.\\
\textbf{Host:} Baoqiang Cao, CTO and Co-founder.\end{tabular}\\\\
Spring 2018 & \begin{tabular}[c]{p{.80\textwidth}}\large\textbf{Undergraduate Research Assistant - \textit{Purdue University}}\\Using Machine Learning models to estimate levels of contact offence through online chat conversations.\\Funded by Office of Undergraduate Research, Purdue University.\\
\textbf{Mentor:} Julia Taylor Rayz.\end{tabular}\\\\
Summer 2017 & \begin{tabular}[c]{p{.80\textwidth}}\large\textbf{Data Scientist Intern - \textit{Perscio}, Indianapolis, IN.}\\ Data Analysis on Healthcare Data.\\Collaboration with SPEA (at Indiana University) to work on Opioid Prescription Trends in Indiana.\\\textbf{Mentor:} Kent Hiller, CTO.\end{tabular}\\\\
Spring 2017 & \begin{tabular}[c]{p{.80\textwidth}}\large\textbf{Undergraduate Research Assistant - \textit{Purdue University}}\\Using Statistical models to understand and predict deviant behavior in the cyberspace.\\
    \textbf{Mentor:} Kathryn Seigfried-Spellar.\end{tabular}
\end{longtable}

\renewcommand*{\arraystretch}{1.5}

\section*{Mentorship}
\vspace{-1em}
\begin{longtable}{p{.09\textwidth}  p{.86\textwidth} }
    \textbf{2018-19} & John Phan (Undergraduate). \textbf{Topic:} \textit{Gender Bias in Word Embeddings}. Awarded NSF REU scholarship. \textbf{Outcome:} Two poster presentations.\\
    % \textbf{2019} & Addison Farinas (Undergraduate). \textbf{Topic:} \textit{Analysis and Annotation of Humorous News Headlines.} \textbf{Outcome:} Humor dataset curation.\\
    \textbf{2020} & Qingyuan ``Carol" Hu and Yi Zhang (Undergraduates). \textbf{Topic:} \textit{Exploring Lexical Irregularities in Hypothesis-only Models of Natural Language Inference.} \textbf{Outcome:} Publication in \textit{IEEE ICCC* CI 2020}, and a presentation at \textit{PURC 2020}, which was \underline{awarded second place} across all students from the Purdue Polytechnic Institute.
    % \textbf{2021} & Sameer Rai Singhal and Priyen Shah (Undergraduates). \textbf{Topic:} \textit{Statistical correlates of World Knowledge in Language Models.}
\end{longtable}

\section*{Reviewing}
% \textbf{\underline{Conference and Journals}}
\vspace{-1em}
\begin{longtable}{p{.15\textwidth}  p{.80\textwidth} }
    \textbf{Primary} & CogSci (2020, 2021, 2022); CoNLL 2021; ACL Rolling Review 2021\\
    \textbf{Secondary} & EMNLP 2020; IJCAI 2020; *SEM 2019; IEEE-IRC 2019.\\
    \textbf{Book} & Chapman \& Hall/CRC Press Statistics Series (2020, 2021).
\end{longtable}

\section*{Service}
\begin{itemize}
    \item \textbf{Local Arrangements Chair}, \textit{Annual Meeting of the North American Fuzzy Information Processing Society 2021 (NAFIPS 2021) held at Purdue University.}
    \item \textbf{Volunteer}, \textit{36th AAAI Conference on Artificial Intelligence}.
    \item \textbf{Program Committee:} CoNLL 2021
    \item \textbf{Graduate Student Advisor}, \textit{Purdue CIT Student Council}.
    \item \textbf{Lab Management}: AKRaNLU lab, Purdue CIT.
    \item \textbf{Organizer}, Undergraduate Research Panel, Purdue CIT.
\end{itemize}
% \begin{itemize}
    % \begin{longtable}{p{.15\textwidth}  p{.85\textwidth} }
    %     \textbf{Reviewing} & CogSci, EMNLP, CRC Press Statistics series, IJCAI, *SEM, IEEE-IRC\\
    %     \textbf{Admin}
    % \end{longtable}
% \end{itemize}

\renewcommand*{\arraystretch}{1}
\section*{Skills}
\vspace{-1em}
\begin{longtable}{p{.25\textwidth}  p{.80\textwidth} }
    \textbf{Programming} & R (expert), Python (expert), SQL (proficient), \LaTeX\\
    \textbf{Libraries} & pytorch, tidyverse(R), tidymodels(R), tensorflow, Rcpp, gensim\\
    \textbf{Statistics} & Probability Theory, GLMs, LMEMs, Bayesian Models\\
    \textbf{Natural~Languages} & English, Hindi, Gujarati, Odiya
    \end{longtable}
% \textbf{\underline{Software Developed}}
\section*{Software Developed}
\paragraph{\href{https://pypi.org/project/minicons/}{\texttt{minicons}}} A toolkit to facilitate behavioral and representational analyses of transformer-based language processing models. \github{https://github.com/kanishkamisra/minicons}
% \begin{itemize}
% \item \textbf{Languages}: Python (proficient), R (proficient), \LaTeX
% \item \textbf{Web Technologies}: HTML, CSS, Javascript (some experience).
% \end{itemize}


% \section*{Natural Languages}
% \begin{itemize}
% \item \textbf{English}: native
% \item \textbf{Mandarin Chinese}: fluent
% \item \textbf{Spanish}: proficient
% \item \textbf{French, German, Modern Standard Arabic}: reading / communicative knowledge
% \end{itemize}

% \section*{Honors and Awards}
% \begin{itemize}
% \item Best Proposal Award, Workshop on Evaluating Vector Space Representations for NLP, Association for Computational Linguistics (ACL) Annual Meeting 2016
% \item National Science Foundation Graduate Research Fellowship recipient, 2014
% \item Member Phi Beta Kappa Academic Honor Society, Brandeis University Chapter
% \item Justice Louis D. Brandeis full-tuition scholarship, Brandeis University, 2006-2010
% \item National Presidential Scholar, U.S. Presidential Scholars Program, 2006
% \end{itemize}

\section*{Professional Affiliations}
\begin{itemize}
    \item Association of Computational Linguistics (ACL)
    \item Cognitive Science Society (CogSci)
    \item Institute of Electrical and Electronic Engineers (IEEE)
    \item Center for Education and Research in Information Assurance and Security (CERIAS)
    \item Society for Mathematical Psychology (MathPsych)
\end{itemize}

\section*{References}
\textbf{NLP Research:} Dr. Julia Taylor Rayz, Dr. Allyson Ettinger, Dr. Victor Raskin\\
\textbf{Teaching:} Dr. Dawn Laux\\
\textbf{Industry:} Dr. Baoqiang Cao, Matt Younkle, Kent Hiller, Bob Boehnlein

% \section*{Teaching}

% \begin{itemize}

% \item \textbf{Computational Linguistics I} 
% \begin{itemize}
% \item Interdisciplinary, mixed undergraduate/graduate level course in computational cognitive modeling and natural language processing (cross-listed in linguistics and computer science)
% \end{itemize}
% \item \textbf{Computational Linguistics II} 
% \begin{itemize}
% \item Advanced continuation of Computational Linguistics I 
% \end{itemize}
% \item \textbf{Seminar in Computational Linguistics}
% \begin{itemize}
% \item Graduate seminar course focusing examining approaches to meaning in computational linguistics---particularly efforts to capture meaning and ``understanding'' in artificial intelligence
% \end{itemize}
% \item \textbf{Meaning in Language: Brains and Machines}
% \begin{itemize}
% \item Interdisciplinary seminar course for advanced undergraduate students from computer science and linguistics (University of Maryland)
% \end{itemize}
% \end{itemize}



\end{document}